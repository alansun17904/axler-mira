\documentclass{article}
\usepackage[utf8]{inputenc}

\usepackage[margin=1in,includefoot]{geometry}
\usepackage{fancyhdr}
\usepackage{mathmacros}

\fancyhead[L]{Measure Theory}
\fancyhead[R]{Alan Sun}
\fancyhead[c]{Chapter 4A: Hardy-Littlewood Maximal Function}
\fancyfoot{}

\fancyfoot[R]{\thepage}
\pagestyle{fancy}

\fancypagestyle{firstpage}{ \fancyfoot[L]{
    \emph{Measure, Integration, and Real Analysis by Sheldon Axler}
}}

\usepackage{tcolorbox}
\tcbuselibrary{breakable}
\usepackage{amsmath}
\usepackage{amssymb}
\usepackage{amsthm}
\usepackage{mathtools}
\usepackage{mathrsfs}
\usepackage{times}
\usepackage{enumitem}
\usepackage{wasysym}
\usepackage{dsfont}

\newcommand{\eps}{\varepsilon}
\theoremstyle{remark}
\newtheorem{claim}{Claim}
\newenvironment{poc}{\textit{Proof of claim:}}{\qed\\}
\newtheorem{theorem}{Theorem}
\newtheorem{lemma}[theorem]{Lemma}

\newlist{legal}{enumerate}{10}
\setlist[legal]{label*=\arabic*.}


\begin{document}
\thispagestyle{firstpage}
\begin{enumerate}[leftmargin=*]
    \item[1.] The proof for this is almost identical to the proof of Markov's
    inequality:
    \begin{align*}
        \mu(\{x \in X: |h(x)| > c\}) &= \frac{1}{c^p} \int_{\{|h(x)| > c\}} c^p \dd\mu, \\
        &\leq \frac{1}{c^p} \int_{\{|h(x)| > c\}} |h(x)|^p \dd\mu, \\
        &\leq \frac{1}{c^p} \| h^p\|_1.
    \end{align*}
    \item[2.] We apply the previous problem, 
    \begin{align*}
        \mu\left(\left\{x \in X : \left|h(x) - \int h\dd\mu\right| \geq c\right\}\right) &\leq \frac{1}{c^2} \int \left(h - \int h \dd\mu \right)^2 \dd\mu, \\
        &= \frac{1}{c^2}\int \left(h^2 - 2h\int h\dd\mu + \left(\int h\dd\mu\right)^2\right) \dd\mu, \\
        &= \frac{1}{c^2} \left(\int h^2 \dd\mu - \left(\int h \dd\mu\right)^2\right).
    \end{align*}
    \item[4.] By way of contradiction, suppose that the constant in the Vitali
    covering lemma can be replaced by $3\eps$ for $0 < \eps < 3$. Then, let $I =
    (a,b)$ such that $b - \frac{3}{2}\eps b - \eps > 0$ and $a < b - \eps$. Now,
    consider $I' = (b-\eps, b-\eps + \ell(I))$. Note that $I$ and $I'$ are not
    disjoint. So, we proceed to show that the Vitali covering lemma no longer
    holds with $3\eps$. It follows that 
    \[
        3\eps \ast (a,b) = \left(a - \frac{3\eps}{2}\ell(I), b + \frac{3\eps}{2}\ell(I)\right).    
    \]
    Then, 
    \begin{align*}
        b + \frac{3\eps}{2} \ell(I) &= b\left(1 + \frac{3}{2}\eps\right) - a, \\
        \intertext{Since $b - \frac{3}{2}\eps b - \eps > 0$, it follows that $2 - \frac{\eps}{b} > 1 - \frac{3}{2}\eps$. Then,}
        &< b\left(2 - \frac{\eps}{b}\right) - a.
    \end{align*}
    The last inequality implies that $3\eps \ast(a,b) \not \supset I'$. By
    symmetry, $3\eps \ast I' \not \supset I$. 
    \item[6.] For any $x \in (0,1)$, there exists an open interval, $I$, such
    that $\bigchi_{[0,1]}(I) = 1$. Thus, $h^\ast((0,1)) = 1$. On the other hand,
    if $x \leq 0$, then $\int_{b-t}^{b+t} |h| > 0$ only if $b+t > 0$. It also
    follows that if $b+t > 1$, then $\int_{b-t}^{b+t} |h|$ decreases
    monotonically. Thus, $h^*(b) = \frac{1}{2(1-b)}$ for $b < 0$. The same holds
    if $b > 1$. So, applying the same logic, $h^*(b) = \frac{1}{2b}$ for $b >
    1$. 
    \item[9.] Let $A = \{b \in \R: h^\ast(b) > c\}$. Then, it suffices to show
    that for every $b \in A$, there exists some $\eps > 0$ such that $(b-\eps,
    b+\eps) \subset A$. We first consider the case of extending $[b,b]$ to the interval $[b,b+\eps)$ and show that 
    there exists an $\eps > 0$ such that $[b,b+\eps) \subset A$, then by symmetry, $(b-\eps,b+\eps) \subset A$. So, 
    \begin{align*}
        \sup_{t > 0} \frac{1}{2t} \int_{b-t}^{b+t} |h| &= 
        \sup_{t >0} \frac{1}{2t}\left(\int_{b-t}^{b+t} |h|- \int_{b-t}^{b+\eps-t} |h| + \int_{b+t}^{b+\eps + t} |h| \right), \\
        &= \sup_{t > 0} \frac{1}{2t} \bigg(\int_{b-t}^{b+t} |h| - \eps \cdot \underbrace{\sup_{(b-t,b+\eps-t)} |h|}_{\text{(a)}} + \eps\cdot\sup_{(b+t,b+\eps+t)} |h|\bigg).
    \end{align*}
    It remains to show that there exists $\eps >0$ where $|h|$ is bounded on $(b-t,b+\eps+t)$. Then, the $\eps$-terms on the right-hand side
    can be made arbitrarily small so, it follows that $[b,b+\eps) \subset A$. 
    By way of contradiction, suppose for all $\eps > 0$, there does not exist some bounded Lebesgue measurable set $L$ such that 
    $(b-t,b+\eps-t) \subset f^{-1}(L)$. Then, there does not exist a bounded Lebesgue measurable set $L$ such that 
    $\cap_{\eps > 0} (b-t,b+\eps-t) \subset f^{-1}(L)$. This implies that $f(b-t) = \infty$ which is a contradiction since 
    $f: \R \to \R$.  
    \item[11.] 
    \item[13.] 
\end{enumerate}
\end{document}
