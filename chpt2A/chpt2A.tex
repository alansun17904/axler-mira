\documentclass{article}
\usepackage[utf8]{inputenc}

\usepackage[margin=1in,includefoot]{geometry}
\usepackage{fancyhdr}
\usepackage{mathmacros}

\fancyhead[L]{Measure Theory}
\fancyhead[R]{Alan Sun}
\fancyhead[c]{Chapter 2A: Outer Measure}
\fancyfoot{}

\fancyfoot[R]{\thepage}
\pagestyle{fancy}

\fancypagestyle{firstpage}{ \fancyfoot[L]{
    \emph{Measure, Integration, and Real Analysis by Sheldon Axler}
}}

\usepackage{tcolorbox}
\tcbuselibrary{breakable}
\usepackage{amsmath}
\usepackage{amssymb}
\usepackage{amsthm}
\usepackage{mathtools}
\usepackage{mathrsfs}
\usepackage{times}
\usepackage{enumitem}
\usepackage{wasysym}

\newcommand{\eps}{\varepsilon}
\theoremstyle{remark}
\newtheorem{claim}{Claim}
\newenvironment{poc}{\\~\textit{Proof of claim:}}{\qed\\}
\newtheorem{theorem}{Theorem}
\newtheorem{lemma}[theorem]{Lemma}

\newlist{legal}{enumerate}{10}
\setlist[legal]{label*=\arabic*.}


\begin{document}
\thispagestyle{firstpage}
\begin{enumerate}[leftmargin=*]
    \item[1.] Since outer measure preserves order and $A\cup B \supset A$, then
    $|A \cup B| \geq |A|$. Now, consider the sequence $A_1, A_2,\ldots$ such
    that $A_1 = A$, $A_2 = B$, and $A_k = \emptyset$ for $k > 2$. It follows
    from the countable subadditivity of outer measure that 
    \begin{align*}
        \left|\bigcup_{k=1}^\infty A_k \right| &= |A \cup B| \\
        &\leq \sum_{k=1}^\infty |A_k| \\
        &= |A| + |B| = |A|. \tag{by given}
    \end{align*} 
    Thus, $|A \cup B| = |A|$.

    \item[2.] We first show that for any $A \subset \R$ that $\ell(tA) =
    |t|\cdot \ell(A)$. If $t=0$, then since $tA = \emptyset$, $\ell(tA) = 0$.
    Then, if $t > 0$, it follows that 
    \begin{equation*}
        tA = \begin{cases}
            (ta, tb) & \text{if}\,A=(a,b), \\ 
            (ta,\infty) & \text{if}\,A=(a,\infty), \\
            (-\infty,ta) &\text{if}\,A=(-\infty,a), \\
            \emptyset & \text{if}\,A=\emptyset.
        \end{cases}
    \end{equation*}
    On the other hand, if $t < 0$, then 
    \begin{equation*}
        tA = \begin{cases}
            (tb, ta) &\text{if}\, A=(a,b), \\
            (-\infty,ta) &\text{if}\, A=(a,\infty), \\
            (ta,\infty) &\text{if}\, A=(-\infty,a), \\
            \emptyset &\text{if}\, A=\emptyset. 
        \end{cases}
    \end{equation*}
    Thus, it follows that $\ell(tA) = |t|\cdot \ell(A)$, by the definition of
    $\ell(\cdot)$. Note that if $t = 0$, then trivially, $|tA| = |\emptyset| =
    0$. So, we only consider the case where $t \neq 0$. Consider a sequence of
    open intervals $I_1, I_2, \ldots$ whose union contains $A$. It follows that
    $tI_1, tI_2, \ldots$ is a sequence of open intervals whose union contains
    $tA$. Then,
    \begin{equation*}
        |tA| \leq \sum_{k=1}^\infty |tI_k| = |t|\sum_{k=1}^\infty |I_k|,
    \end{equation*}
    by taking the \textit{infimum} over all such sequences, we have that $|tA|
    \leq |t|\cdot |A|$. Now, observe that $|\frac{1}{t} (t A)| = |A|$, so 
    \begin{equation*}
        |t| \left|\frac{1}{t} (tA)\right| \leq |t| \sum_{k=1}^\infty \ell(\frac{1}{t}(tA)) 
        \leq \frac{|t|}{|t|}\sum_{k=1}^\infty \ell(tI_k) = \sum_{k=1}^\infty |tI_k|.
    \end{equation*}
    Again, by taking the \textit{infimum} over all such sequences we have that
    $|t|\cdot |A| \leq |tA|$. Thus, $|tA| = |t|\cdot |A|$. 

    \item[3.] It suffices to show that $|B\setminus A| + |A| \geq |B|$. By the
    subadditivity of outer measure, $|B\setminus A| + |A| \geq |(B\setminus A)
    \cup A|$. Since $(B \setminus A)\cup A \supset B$ and outer measure
    preserves order then $|(B\setminus A) \cup A| \geq |B|$. Thus, $|B\setminus
    A| \geq |B| - |A|$. 

    \item[9.] Suppose that $A$ is bounded, then there exists some $x, r \in \R$
    such that $A \subset (x-r, x+r)$. Then, for any $t > \max\{|x-r|, |x+r|\}$,
    it follows that $A \cap (-t, t) = A$. Thus, the limit expression holds when
    $A$ is bounded. Now, consider the case where $A$ is unbounded. First, the
    limit must exist since for any sequence of $t_k \to \infty$, it follows that
    $|A \cap (-t_k,t_k)|$ is monotone. Moreover, it must be that $|A| \geq
    \lim_{t\to\infty} |A\cap (-t,t)|$ since for any $t > 0$, $|A| \geq |A \cap
    (-t,t)|$. So, it remains to show that $|A| \leq \lim_{t \to \infty} |A \cap
    (-t, t)|$. 

    

    \item[11.] By the definition of outer measure, since


    \item[12.] \begin{enumerate}[label=(\alph*)]
        \item The arbitrary union of open sets is open. Thus,
        $\bigcup_{k=1}^\infty (r_k - 1/2^k, r_k + 1/2^k)$ is open. And, by
        definition, the complement of an open set is closed. So, $F$ is closed. 
        \item By way of contradiction, suppose that there exists an interval $I
        \subset F$ such that $I$ contains more than one element. That is, $I =
        (a,b)$ for $a < b$. Then, it follows from the denseness of $\Q$ in $\R$
        that there exists a rational number in $I$. Thus, there is an centered
        around this rational number that is not contained in $F$. Thus, $I$ is
        not connected and therefore, $I$ is not an interval. So, any intervals
        in $F$ contain only one element. 
        \item It follows from problem 3 that 
        \begin{align*}
            |F| &\geq |\R| - \left|\bigcup_{k=1}^\infty \left(r_k - \frac{1}{2^k}, r_k + \frac{1}{2^k}\right)\right|, \\
            &\geq |\R| - \sum_{k=0}^\infty \frac{1}{2^k}, \\
            &\geq |\R| - 2, \\
            &\geq \infty.
        \end{align*}
    \end{enumerate}
\end{enumerate}
\end{document}
