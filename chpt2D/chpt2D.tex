\documentclass{article}
\usepackage[utf8]{inputenc}

\usepackage[margin=1in,includefoot]{geometry}
\usepackage{fancyhdr}
\usepackage{mathmacros}

\fancyhead[L]{Measure Theory}
\fancyhead[R]{Alan Sun}
\fancyhead[c]{Chapter 2D: Lebesgue Measure}
\fancyfoot{}

\fancyfoot[R]{\thepage}
\pagestyle{fancy}

\fancypagestyle{firstpage}{ \fancyfoot[L]{
    \emph{Measure, Integration, and Real Analysis by Sheldon Axler}
}}

\usepackage{tcolorbox}
\tcbuselibrary{breakable}
\usepackage{amsmath}
\usepackage{amssymb}
\usepackage{amsthm}
\usepackage{mathtools}
\usepackage{mathrsfs}
\usepackage{times}
\usepackage{enumitem}
\usepackage{wasysym}

\newcommand{\eps}{\varepsilon}
\theoremstyle{remark}
\newtheorem{claim}{Claim}
\newenvironment{poc}{\textit{Proof of claim:}}{\qed\\}
\newtheorem{theorem}{Theorem}
\newtheorem{lemma}[theorem]{Lemma}

\newlist{legal}{enumerate}{10}
\setlist[legal]{label*=\arabic*.}


\begin{document}
\thispagestyle{firstpage}
\begin{enumerate}[leftmargin=*]
    \item[5.] From the equivalent definitions of the Lebesgue measurable sets,
    there must exist a sequence of closed sets $F_1, F_2, \ldots$ where $F_k
    \subset A$ and 
    \[
        \left|A \setminus \bigcup_{k=1}^{\infty} F_k\right| = 0.
    \]
    Since any finite union of closed sets is closed, then it follows that $F_1,
    F_1 \cup F_2, F_1 \cup F_2 \cup F_3, \ldots$, is a sequence of increasing
    closed sets also contained in $A$. And since $\bigcup_{k=1}^\infty F_k =
    \bigcup_{k=1}^\infty \bigcup_{\ell=1}^k F_\ell$, then the previous equation
    holds.
    \item[6.] Let us first assume that $A$ is a Lebesgue measurable set. Then,
    first consider the following claim:
    \begin{claim}\label{claim:closed-bounded}
        If $|A| < \infty$, then for every $\eps >0$, there exists a closed,
        bounded set $F \subset A$ such that $|A \setminus F| < \eps$. 
    \end{claim}
    \begin{poc}
        By problem 9 in section 2A, for any set $A \subset \R$, it follows that 
        \[
            \lim_{t\to\infty} |(-t,t) \cap A| = |A|.    
        \]
        From the definition of the limit, for any $\eps > 0$, there exists some
        $t\in\R$ such that $|A \setminus \left((-t,t)\cap A\right)| = |A| -
        |(-t, t)\cap A| < \eps$. For any $t \in \R$, $(-t, t) \cap A$ is
        bounded. Moreover, $(-t, t) \cap A$ must be a Lebesgue measurable set
        (since every Borel set is also a Lebesgue measurable set). Since $(-t,
        t) \cap A$ is a Lebesgue measurable set, there must exist a closed set
        $F \subset (-t, t) \cap A$ such that $|(-t, t) \cap A \setminus F| <
        \eps$. Then, by countable additivity, it must be that $|A \setminus F| =
        |A \setminus (-t, t)\cap A| + |(-t, t) \cap A \setminus F| < 2\eps$.
    \end{poc}
    Now, we show the forward claim:
    \begin{claim}
        If $A$ is Lebesgue measurable $|A| < \infty$, then for every $\eps > 0$,
        there exists $G$, a union of finite, disjoint, bounded open intervals
        such that $|G \Delta A| < \eps$. 
    \end{claim}
    \begin{poc}
        % Recall that if $A$ is Lebesgue measurable, then for all $\eps > 0$,
        % there exists an open set $G \supset A$ such that $|G \setminus A| <
        % \eps$ and by assumption $|A \setminus G| = 0$. Since $G$ is open, it
        % must be the disjoint union of countable open intervals: $G_1, G_2,
        % \ldots$. Note that any $G_k$ must be bounded otherwise, $|A| \geq
        % \infty$.  
        % And, there exists a closed set $F \subset A$ such that $|A \setminus F|
        % < \eps$. Now, it remains to show that we can actually construct a finite
        % union of these open intervals.

        By claim~\ref{claim:closed-bounded}, it follows that there exists a closed, bounded set $F \subset A$
        such that $|A \setminus F| < \eps$. Since $F$ is a Borel set, there must be a open cover $G \supset F$
        such that $|G \setminus F| < \eps$. $G$ is the union of finite disjoint open intervals $G_1, G_2,\ldots$,
        by the Heine-Borel theorem, there exists a finite subcover of $G' = \bigcup\{G_k\}_{k=1}^n$ of these open intervals. 
        Note that because $F$ is bounded and $|G\setminus F| < \eps$ each of these open intervals must also be bounded.
        Since $F\subset G' \subset G$, it must be that $|G' \setminus F| < \eps$. So,
        \begin{align*}
            |G' \Delta A| &= |G' \setminus A| + |A \setminus G'|, \\
            &\leq |G' \setminus F| + |A \setminus F|, \\
            &< 2\eps.
        \end{align*}
    \end{poc}
    Now, for the converse:
    \begin{claim}
        By the definition of outer measure, there exists a sequence of open intervals $\{I_k\}_{k=1}^\infty$ 
        such that $|\bigcup_{k=1}^\infty I_k| < |A\setminus G| + \eps$. Thus, it follows that $G\cup \bigcup_{k=1}^\infty I_k$
        is open and $|G\cup \bigcup_{k=1}^\infty I_k| < |A| + 2\eps$.  
    \end{claim}
    \item[12.] The forward is trivial and follows directly from the definition of the Lebesgue measure 
    since $A, (b,c) \setminus A$ are disjoint. Now, we show the converse directly,
    \begin{claim}
        If $A \subset (b,c)$ and $|A| + |(b,c) \setminus A| = c - b$, then $A$ is Lebesgue measurable.
    \end{claim}
    \begin{poc}
        By the definition of outer measure, there exists a sequence of open intervals $\{I_k\}_{k=1}^\infty$
        such that $|\bigcup_{k=1}^\infty I_k| < |(b,c)\setminus A| + \eps$. Now consider $B = [b+\eps,c-\eps] \setminus \bigcup_{k=1}^\infty I_k$.
        such that $\eps < \min\{b,c\}$. It follows that $B$ is a closed subset of $A$. It remains to show that $A \setminus B$ can
        be made arbitrarily small. By order-preservation of outer measure $|B| \leq |A|$ and since $B$ is closed, 
        \begin{align*}
            |A \setminus B| &= |A| - |B|, \\
            |B| &> (c-b+2\eps) - \left(|(b,c)\setminus A| + \eps\right), \\
            &= (c-b+2\eps) - \left((c-b) - |A| + \eps\right), \\
            &= |A| + \eps.
        \end{align*}
        Then, it follows that $|A| - |B| < \eps$. So, $A$ is Lebesgue measurable.
    \end{poc}
    \item[13.] It must be that $(-n,n) \cap A \subset (-n,n)$. So, let $B = (-n,n) \cap A$. Then,
    $(-n,n) \setminus A = (-n,n) \setminus B$. Since 
    \begin{align*}
        x \in (-n,n) \setminus A &\iff x \in (-n,n) \,\text{and}\, x\notin \setminus A, \\
        &\iff x \in (-n,n) \,\text{and}\, x \notin A \cap (-n,n), \\
        &\iff x \in (-n,n)\setminus B.
    \end{align*}
    So, by problem 12 in this section, this equality only holds if and only if $(-n,n) \cap A$ is Lebesgue 
    measurable for all $n\in \Z^+$. Thus, in the limit as $n\to\infty$, the equality only holds as $A$ is 
    Lebesgue measurable.
    \item[24.] \begin{enumerate}[label=(\alph*)]
        \item Follows directly from claim~\ref{claim:closed-bounded}. Since this holds for every $\eps > 0$, and the 
        limit in claim~\ref{claim:closed-bounded} is monotonically increasing, then it must exist and so 
        $\sup_{t\in\R} |F| = |A|$. 
        \item Consider the Vitali set.
    \end{enumerate}
\end{enumerate}
\end{document}
