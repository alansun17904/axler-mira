\documentclass{article}
\usepackage[utf8]{inputenc}

\usepackage[margin=1in,includefoot]{geometry}
\usepackage{fancyhdr}
\usepackage{mathmacros}

\fancyhead[L]{Measure Theory}
\fancyhead[R]{Alan Sun}
\fancyhead[c]{Chapter 2C: Measures and Their Properties}
\fancyfoot{}

\fancyfoot[R]{\thepage}
\pagestyle{fancy}

\fancypagestyle{firstpage}{ \fancyfoot[L]{
    \emph{Measure, Integration, and Real Analysis by Sheldon Axler}
}}

\usepackage{tcolorbox}
\tcbuselibrary{breakable}
\usepackage{amsmath}
\usepackage{amssymb}
\usepackage{amsthm}
\usepackage{mathtools}
\usepackage{mathrsfs}
\usepackage{times}
\usepackage{enumitem}
\usepackage{wasysym}

\newcommand{\eps}{\varepsilon}
\theoremstyle{remark}
\newtheorem{claim}{Claim}
\newenvironment{poc}{\textit{Proof of claim:}}{\qed\\}
\newtheorem{theorem}{Theorem}
\newtheorem{lemma}[theorem]{Lemma}

\newlist{legal}{enumerate}{10}
\setlist[legal]{label*=\arabic*.}


\begin{document}
\thispagestyle{firstpage}
\begin{enumerate}[leftmargin=*]
    \item[5.] 
    \item[10.] Let $X = \R$, $\cS$ be the Borel sets, and let $\mu$ 
    be the Lebesgue measure. Let $E_k$ be the following set:
    \[
        E_k = \bigcup_{x \in \Q} \left(x-\frac{1}{k}, x + \frac{1}{k}\right).
    \]
    Then, it follows that for any $E_k, \mu(E_k) = \infty$. Since $\bigcap_{k=1}^\infty E_k = \Q$, 
    $\mu\left(\bigcap_{k=1}^\infty E_k\right) = 0$. However, $\lim_{k\to\infty} \mu(E_k) = \infty$. 
    Thus, the hypothesis that $\mu(E_1) < \infty$ is necessary.
\end{enumerate}
\end{document}
